%%%%%%%%%%%%%%%%%%%%%%%%%%%%%%%%% MAIN FINDINGS
\subsection{Main Findings}
We fit our Bayesian model with uniform prior via {\tt Stan} using $10$ chains of $10,000$ iterations each. Table \todo{XXXX} reports the mean of the posterior parameters, as long as the associated 95\% confidence intervals, of the DDE and PCB variables and their interaction with race group. All the other coefficients, including the center effects, are reported in the appendix. Two main facts are worth underlining. First, we see that the estimates are in line with what we expected: an increase in either the  PCB or DDE concentration in blood leads to higher odds of having a dangerous delivery. In particular, a 1 unit increase of $\text{DDE}_{\text{exposure}}$ increments the odds of a more adverse preterm by $2.02\%$ for non whites and by $7.25\%$ for white women\footnote{Say that $\beta$ is our coefficient associated to the variable $x$ in the ordinal logistic regression. Then, if the outcomes are ordered from worst to best (like in our example), an increase in 1 unit of $x$ is associated with a variation of the odds of the worst outcome by $(e^{\beta}-1)*100\%. }. On the other hand, increments by 0.1 units of $\text{PCB}_{\text{exposure}}$ increases the odds of the worst delivery by $19.22\%$ for non-white women, and by $1.595\%$ for white ones. The above percentages highlight the second noticeable fact in our analysis, that is the race dependency of the exposures. This fact is particularly evident from figure \todo{XXX}, which plots how the predicted probabilities of each delivery group vary for higher levels of PCB and DDE\footnote{The curves are computed with reference to center "5" and for non-smoking individuals. Probabilities for DDE variations are predicted holding PCB constant a his mean, and viceversa.}. We see that, as the adjusted exposures increase, the probability of delivering at term decreases (with whites more sensitive to DDE and non-whites to PCB).
%%%%%%%%%%%%%%%%%%%%%%%%%%%%%%%%% SENSITIVITY ANALYSIS
\subsection{Sensitivity Analysis}
To further check the robustness of our finding, we test the model with a different prior and varying hyperparameters. In particular, we set an $R^2$ prior with location at $0.3$, $0.5$ and $0.8$ respectively. Table \ref{tab:priors}) reports the estimated means of the posterior coefficient and the confidence intervals. We see that the effects do not vary significantly in magnitude. For example, \todo{Add a little interpretation as well, maybe for one group}. To conclude, our method is not sensitive to the choice of priors.
